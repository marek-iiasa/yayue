\documentclass[a4paper,12pt]{article}
\usepackage{fullpage,times}
\usepackage{verbatim}  % for comments
% \usepackage{euro}
% \EUR~10.10

%\input{Figs}
\input mm_macros.tex

\DeclareMathAlphabet{\mathbit}{OT1}{cmr}{bx}{it}
\def\vv#1{\ensuremath{\mathbit{\bf #1}}}

\usepackage{amsmath}
\usepackage{lineno}
% \usepackage{showlabels}

\pagestyle{plain}
\thispagestyle{plain}


\begin{document}
\centerline{\Large\bf SMS of simplified cn\_liquid model}
\bigskip

\centerline{\sc version 1.0 of \today}

\bigskip
\bigskip

\linenumbers

{\em Note:} The SMS documented here is based on the SMS of the cn\_ptl model of
Sep.~17, 2021 (see the cn-ptl repo, ..../prototype/docs).
This SMS has been implemented in pyomo as the {\em cn\_liquid v.1.0} model.
Whenever modifications of the model will be desired, this document will be kept
consistent with the pyomo model implementation.

\section{Symbolic model specification (SMS)}
The SMS documents all model entities (indexing structure, variables,
parameters, relations).
This draft presents SMS of incomplete model prototype being implemented in Pyomo
for testing Pyomo implementation and reporting module.
The prototype will be gradually extended and use in the corresponding MCMA.

\subsection{The model purpose}
The model aims at supporting analysis of the relations between the decisions
on the technology portfolio (investment for capacity expansion and activity-levels)
for producing liquid fuels, and the consequences of implementation of such decisions.
The consequences are measured by values of the outcome variables that represent
conflicting objectives, such as costs and externalities.

\subsection{Indexing structure}
\begin{comment}
\subsubsection{Intro to indexing structure}
The SMS uses the Structured Modeling (SM) concepts; in particular the
compound entities.
For example, a compound variable $\vv{x}$ actually represents a set of variables
$x_{ij}, i \in I,\; j \in J$, where $i$ and $j$ stand for indices, and $I$ and
$J$ are the sets of values of the corresponding indices.
To illustrate this concept, let $\vv{x}$ be flows between $i$-th warehouse
and $j$-th store.
Then the set $I$ of warehouses can be defined as $I = \{city1, city2, \dots\}$.
Similarly, the set $J$ of shops can be defined as $J = \{loc1, loc2, \dots\}$.

Thus, the indexing structure is composed of:
\btlb
\item symbols of indices (typically a lower-case letter), and
\item symbols of sets (typically, the corresponding upper-case letter).
\etl
The examples of the index-sets below are for illustration only.
The actual members of these sets are defined by the model parameters.
\end{comment}

\subsubsection{Model configuration}\label{sec:cfg}
Model instances corresponding to this SMS are generated based on the values
in a data set, except of two configuration parameters defined in the SMS, namely:
\btlb
\item {\em periods}: number of planning periods, and
\item {\em lifet}: number of planning periods in which the newly installed
	capacity remains available.
\etl

\subsubsection{Indexing structure of the model}\label{sec:index}
The model uses the following indices and the corresponding sets:
\btlb
\item $p \in P$: the ids of the 5-year planing periods; e.g., for
	{\em periods = 7}, $P = \{0, 1, 2, \dots, 6\}$.\footnote{
	Instead of using the calendar-year values, we use sequence of positive integer
	values, i.e., for the calendar years \{2020, 2025, \dots, 2050\} we index
	the planning periods by non-negative integer numbers.
	The correspondence between the planning periods and the calendar years
	can be defined (for reporting) by a simple mapping.}
\item $H$: set of previous, immediately before the first planning\footnote{
	First planning period is indexed by 0.} periods.
	If the newly installed capacity is available only in the vintage period,
	i.e., $lifet = 0$, then $H$ is an empty set.
	For $lifet > 0$ it is defined by $H = RangeSet(-lifet, -1)$.\footnote{
	Note the difference between $RangeSet(1, 3)$ and the sequence returned by
	the standard Python function $range(1, 3)$. The first one has three elements,
	while the second has only two elements.}
	For example, for $lifet = 3$, $H = \{-3, -2, -1\}$.
\item $v \in V,\; V = H \cup P$: vintage period (i.e., period in which the newly
	installed capacity becomes available).
	% The duration (consecutive periods during which the capacity remains available)
	% of each newly installed capacity is defined by the parameter~$lt$
	% (life-periods);
	In other words, in a current period the available capacities consists of:
	\btlbs
	\item new capacity installed in the current period, and
	\item new capacities installed in each of the $lifet$ periods immediately
		before the current period.
	\etls
	Therefore, negative indices of $V$ correspond to historical periods
	(prior to the first planning period); the capacities installed in these periods
	can still be used during the planning periods.
\item $(p, v) \in V_p$ (the set is denoted in the pyomo implementation by VP):
	two-dimensional set of all feasible pairs of $p$ and $v$ indices.
\item $t \in T$ technologies: e.g., $T = \{ {\rm OTL, CTL, BTL, PTL} \}$ or
	$T = \{ {\rm OTL, PTL} \}$.
	% $T_f \subset T$ technologies directly producing the final commodities
\item $f \in F$: final commodities (products), i.e., liquid fuel.
	E.g., $F = \{ {\rm gasoline, diesel-oil} \}$ (if demand is defined for each
	type of liquid fuel) or
	$F = \{ {\rm fuel} \}$ (if demand is defined for an aggregate of all
	considered fuels).
\begin{comment}
\item $c \in C$: commodity. E.g.,
	$C = \{ {\rm oil, gasoline, coal, crude-oil, \dots } \}$.
	Commodities belong to diverse subsets that correspond to their roles:
	\btlas{$\diamond$}
	\item $C_f \subset C$: final commodities (currently: gasoline and diesel-oil),
	\item $C_i \subset C$: commodities required as inputs for activities of
		technologies~$t \in T$,
	\item \dots  (more subsets shall be defined).
	\etls
\end{comment}
\etl

\subsection{Variables}
Although all variables are treated equally within the model,
we divide the set of all model variables into categories corresponding to the roles;
this helps for structuring the model presentation.

\subsubsection{Decision variables}
The technology portfolio is implied by values the decision (control) variables:
\btlb
\item $ncap_{tv}, t \in T, v \in P$: new production capacity of $t$-th
	technology, made available at the beginning of $v$-th period.\footnote{
	We use single-letter indices, which allows for the compact notation,
	i.e., $ncap_{tv}$ instead of $ncap_{t,v}$.}
	The capacity that is "new", and available, in $v$-th period
	remains available in $lifet$) following periods.\footnote{In former model
	versions life-periods $lt$ were denoted by~$\tau$.}
	Note that the capacities installed in historical periods are defined by
	the model parameters.
\item $act_{tpv}, t \in T, (p, v) \in V_p$: activity level of $t$-th
	technology, using in period $p$ the new capacity provided in period~$v$.
\etl

\subsubsection{Outcome variables}
Outcome variables are used for evaluation of the consequences of implementation
of the decisions; therefore, at least one of them is used as the optimization
objective.

In the model prototype only three outcomes (to be used used as criteria in
multiple-criteria model analysis) are defined:
\btlb
\item $cost$: the total cost of the system over the planning period,
\item $invCost$: the investment costs of new capacities, and
\item $omCost$: the O\&M (operations and maintainance) costs, and
\item $co2$: the total CO2 emission caused by the production system.
\etl

\begin{comment}
\subsubsection{State variables}
The variables defining the state of the system:
\btlb
\item \dots (to be defined, if needed).
\etl
\end{comment}

\subsubsection{Auxiliary variables}
All other variables used in the SMS:
\btlb
\item $omcVar$: variable part of $omCost$
\item $omcFixP$: fixed part of $omCost$ related to the capacities installed in
	planning periods,
\item $omcFixH$: fixed part of $omCost$ related to the capacities installed
	before planning periods,
\item $omcFix$: total ($omcFixP + omcFixH$) fixed part of $omCost$
% \item $actInp_{cp}$ input resources required by all technologies $t \in T$
\etl

\subsection{Parameters}
The following model parameters are used in the model relations 
specified in in Section~\ref{sec:rel}:
\btlb
\item $dis_p$: discount rate
\item $hncaph_{tv}, v \in H$: new capacities installed in historical
	periods,\footnote{For the sake of brevity, iterators $f \in F, p \in P$,
	and $t \in T$ are further on omitted, whenever these are obvious.}
\item $d_{fp}$: demand for final commodities defined (fuel),
\item $a_{tvf}, v \in V$: amount of output from the unit of the corresponding activity,
\item $cuf_{tv}, v \in V$: capacity utilization factor
\item $invC_{tv}, v \in P$: unit investment cost of new capacity
\item $vom_{tv}, v \in V$: variable O\&M costs,
\item $fom_{tv}, v \in V$: fixed O\&M costs,
\item $ef_{tv}, v \in V$: CO2 emission factor

\begin{comment}
\item $inp_{ctv}$: amount of input required by the unit of the corresponding activity
\item $omcC_{tv}$: unit OMC of activity
\item $extP_{cp}$: unit price of external input resources
\end{comment}
\etl

\subsection{Relations}\label{sec:rel}
The values of the model variables conform to the following model relations.
\btla{88.}
\inum The sum of activities $act_{tvp}$ shall result in producing the required
	amounts of the final commodities:
	\be
	\sum_{t \in T} \sum_{v \in V_p} a_{tvf} \cdot act_{tpv} \ge d_{fp}, \quad
		f \in F,\; p \in P,\; (p, v) \in V_p.
	\ee
\inum The levels of activities cannot exceed the corresponding available
	capacities. This relation holds for two situation, namely, when the
	capacity is defined by either the decision variable (then $v \ge 0$)
	or by the $hncap$ parameter ($v < 0$), respectively.
	Therefore, the corresponding relation is generated in either~\eqref{eq:avncap}
	or~\eqref{eq:avhcap} form:
	\be\label{eq:avncap}
		act_{tpv} \le cuf_{tv} \cdot ncap_{tv}, \quad t \in T,\; p \in P,\;
		(p, v) \in V_p,\; v \ge 0,
	\ee
	\be\label{eq:avhcap}
		act_{tpv} \le cuf_{tv} \cdot hncap_{tv}, \quad t \in T,\; p \in P,\;
		(p, v) \in V_p,\; v < 0.
	\ee
\inum Investment costs of new capacities are defined by:
	\be
		invCost = \sum_{t \in T} \sum_{v \in P} dis_p \cdot invC_{tv} \cdot ncap_{tv}
	\ee
\inum Operations and maintenance costs, defined by~\eqref{eq:omCost}, consists of 
	two parts, namely the variable costs~\eqref{eq:omcvar}, and the fixed
	(implied by the installed capacity) costs, see~\eqref{eq:omcfx}:
	\be\label{eq:omCost}
		omCost = omcFix + omcVar
	\ee
	\be\label{eq:omcvar}
		omcVar = \sum_{t \in T} \sum_{p \in P}\; dis_p \sum_{(p, v) \in V_p}
			vom_{tv} \cdot act_{tpv}
	\ee
	\be\label{eq:omcfx}
		omcFix = omcFixP + omcFixH
	\ee
	The fixed O\&M costs are related to the capacities installed in the planning
	periods (the model variable) and to the capacities installed in historical
	periods (which are given).
	These two components are represented by~\eqref{eq:omcfxp} and~\eqref{eq:omcfxh},
	respectively.
	\be\label{eq:omcfxp}
		omcFixP = \sum_{t \in T} \sum_{p \in P} dis_p \cdot fom_{tp} \cdot ncap_{tp},
	\ee
	\be\label{eq:omcfxh}
		omcFixH = \sum_{t \in T} \sum_{p \in P} dis_p \sum_{v \in H}
			fom_{tv} \cdot hncap_{tv}.
	\ee
\begin{comment}
\inum Each activity requires the corresponding input resources.
	The following relation defines the inputs of the corresponding commodities
	required as inputs by all technologies:
	\be
		actInp_{cp} = \sum_{t \in T} \sum_{v \in V_p} inp_{ctv} \cdot act_{tvp},
		\quad  c \in C_i,\; p \in P
	\ee
\inum Costs of externally provided inputs are defined by:
	\be
		extCost = \sum_{c \in C_i} \sum_{p \in P} extP_{cp} \cdot actInp_{cp}
	\ee
\end{comment}
\inum Total cost is defined by:\footnote{Costs of inputs shall be added after
	clarification how it relates to the $omCost$.}
	\be
		% cost = invCost + omCost + extCost
		cost = invCost + omCost
	\ee
\inum The total CO2 emission caused by the activities is defined by:
	\be\label{eq:co2}
		co2 = \sum_{t \in T} \sum_{p \in P} \sum_{v \in V_p} ef_{tv} \cdot act_{tvp}
	\ee
\etl
{\em Note:} entities describing the up-stream technologies should be
researched and added to the SMS in a future version of the model.
Then eq.~\eqref{eq:co2} shall be modified accordingly.

\end{document}

